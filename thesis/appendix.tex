%%****************************************************************************
%** Copyright 2005 by Bernhard Tellenbach, <bernhard.tellenbach@airmail.ch>
%** Information is provided under the terms of the
%** GNU Free Documentation License <http://www.gnu.org/copyleft/fdl.html>
%****************************************************************************
%****************************************************************************
%** Last Modification: 2005-07-11 1600
%** 2005-07-11	Updated the syntax to match the current nomencl packet
%****************************************************************************

\chapter{\label{appendixMessagingProtocol}Relay Messaging Protocol}
The following table shows the exact specification of the different message types used in the \textit{relay protocol}. The field sizes and the communication directions are given in the corresponding columns.

\begin{center}
\begin{tabular}{|l|l|l|l|}
	\hline
	\textbf{Message Type} & \textbf{Communication} & \textbf{Size (bits)} & \textbf{Field} \\
	\hline
	REY	& C $\leftrightarrow$ S 	& 24		& Command (=REY) \\
		& 		& 4		& Flag \\
		& 		& 20		& Secret \\
	\hline
	INV	& M $\to$ C 	& 24		& Command (=INV) \\
		& 		& 256	& Block hash \\
		& 		& 16	& Segment count \\
		\hline
	SEG	& C $\to$ S 	& 24	& Command (=SEG) \\
		& 				& 256	& Block hash \\
		& 				& 16	& Segment id \\
		\hline
	BLK	& S $\to$ C 	& 24	& Command (=BLK)\\
		& 				& 16	& Segment id \\
		& 				& 499*8	& Segment data \\
		\hline
	BLK	& M $\to$ S 	& 24	& Command (=BLK)\\
		& 				& 16	& Segment id \\
		& 				& 499*8	& Segment data \\
		&				& 16	& precomputed UDP checksum \\
\hline
	ADV	& C $\to$ S 	& 24	& Command (=ADV)\\
		& 				& 256	& Block hash \\
\hline
	CTR	& S $\to$ C 	& 24	& Command (=CTR)\\
		& 				& 32	& IP \\
		& 				& 16	& port \\
		&				& 8		& empty \\
\hline
	CON	& S $\to$ M 	& 24	& Command (=CON) \\
		& 				& 8		& empty \\
		& 				& 16	& port \\
		& 				& 32	& IP \\
\hline
	UPD	& M $\to$ S 	& 24	& Command (=UPD) \\
		& 				& 256	& Block hash \\
\hline
	BCL	& M $\to$ S 	& 24	& Command (=BCL) \\
		& 				& 32	& ip \\
	\hline
\end{tabular}\\
\end{center}
M: Controller\\
S: Switch\\
C: Client\\




\chapter{\label{appendix:pingClient}Ping Client Modifications}
The following code alterations are made to create the \textit{ping client} from the regular bitcoin client. 


\begin{diffCode}
diff --git a/src/net.h b/src/net.h
index 8378a303b..cd1dda236 100644
--- a/src/net.h
+++ b/src/net.h
@@ -38,7 +38,7 @@ class CScheduler;
 class CNode;

 /** Time between pings automatically sent out for latency probing and keepalive (in seconds). */
-static const int PING_INTERVAL = 2 * 60;
+static const int PING_INTERVAL = 1;
 /** Time after which to disconnect, after waiting for a ping response (or inactivity). */
 static const int TIMEOUT_INTERVAL = 20 * 60;
 /** Run the feeler connection loop once every 2 minutes or 120 seconds. **/
diff --git a/src/net_processing.cpp b/src/net_processing.cpp
index bf9307727..de2ced1ae 100644
--- a/src/net_processing.cpp
+++ b/src/net_processing.cpp
@@ -2720,6 +2720,7 @@ bool static ProcessMessage(CNode* pfrom, const std::string& strCommand, CDataStr

     else if (strCommand == NetMsgType::PONG)
     {
+	LogPrintf("THROUGHPUT: PONG: %u\n", GetTimeMicros());
         int64_t pingUsecEnd = nTimeReceived;
         uint64_t nonce = 0;
         size_t nAvail = vRecv.in_avail();
@@ -3175,6 +3176,7 @@ bool PeerLogicValidation::SendMessages(CNode* pto, std::atomic<bool>& interruptM
         }
         if (pto->nPingNonceSent == 0 && pto->nPingUsecStart + PING_INTERVAL * 1000000 < GetTimeMicros()) {
             // Ping automatically sent as a latency probe & keepalive.
+	    LogPrintf("THROUGHPUT: PING: %u\n", GetTimeMicros());
             pingSend = true;
         }
         if (pingSend) {
@@ -3194,6 +3196,8 @@ bool PeerLogicValidation::SendMessages(CNode* pto, std::atomic<bool>& interruptM
             }
         }

+	return true;
+
         TRY_LOCK(cs_main, lockMain); // Acquire cs_main for IsInitialBlockDownload() and CNodeState()
         if (!lockMain)
             return true;	
\end{diffCode}












%Entries for the list of abbrevations:
%
%To generate the list of abbrevations, execute:
%makeindex Thesis.nlo -s nomencl.ist -o Thesis.nls
%
%If you are using TeXniCenter, specify:
%"%bm.nlo" -s nomencl.ist -o "%bm.nls"
%as beeing the argument list for makeindex.
%---------------------------------------------------------------------------------------------------------
%For old nomencl package uncomment this:
%\printglossary
%For new nomencl package uncomment this:
\printnomenclature

\abbrev{XCA}{\markup{X}tremely \markup{C}ool \markup{A}bbrevations}


