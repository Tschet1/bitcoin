%****************************************************************************
%** Copyright 2001, 2002, 2003, 2004 by Lukas Ruf, <lukas.ruf@lpr.ch>
%** Information is provided under the terms of the
%** GNU Free Documentation License <http://www.gnu.org/copyleft/fdl.html>
%** Fairness: Cite the source of information, visit <http://www.topsy.net>
%****************************************************************************
%** Last Modification: 2013-07-31
%** 2004-02-17: Lukas Ruf
%**             Added recommendation by Thomas Duebendorfer
%**             Added different babel languages
%**             Added more comments
%** 2004-10-16: Lukas Ruf
%**             More comments
%**             Added subfigure
%** 2005-07-11	Bernhard Tellenbach
%**							Added \abbrev command to generate a list of abbrevations
%**							Removed support for psfig and epsfig (old)
%** 						Adapted syntax for new nomencl packet version
%** 2013-07-31 David Gugelmann (gugdavid)
%**            Adapted for pdflatex
%**            Added hyperref
%**            Style of headers modified, adapted watermark command
%****************************************************************************

\RequirePackage{times}

\usepackage[english]{babel}
%-% \usepackage[german]{babel}
%-% \usepackage[ngerman]{babel}

\usepackage[utf8]{inputenc}
\usepackage[T1]{fontenc}
\usepackage{type1cm}

\usepackage{a4}

\usepackage{hyperref}
\hypersetup{colorlinks,%
            citecolor=black,%
            filecolor=black,%
            linkcolor=black,%
            urlcolor=black,%
            pdftex}

\usepackage[pdftex]{color,graphicx} % pdftex does not read eps files -> use epstopdf to convert files
\graphicspath{{Figures/},{logos/}}

\usepackage{caption}
\usepackage{subcaption} % caption/subcaption replaces subfigure, which is deprecated

\usepackage{fancyhdr}
\usepackage{fancybox}

\usepackage{float}
\usepackage{longtable}
\usepackage{paralist}
\usepackage{url}
%\usepackage{portland}
\usepackage{lscape}
\usepackage{moreverb}

\usepackage{nomencl}
  \let\abbrev\nomenclature
  \renewcommand{\nomname}{List of Abbrevations}
  \setlength{\nomlabelwidth}{.25\hsize}
  \renewcommand{\nomlabel}[1]{#1 \dotfill}
  \setlength{\nomitemsep}{-\parsep}
  %For old nomencl package, uncomment this:
  \makeglossary 
  %For new nomencl package, uncomment this:
  %\makenomenclature

\usepackage[normalem]{ulem}
  \newcommand{\markup}[1]{\uline{#1}}
  
%% Thanks to Thomas Duebendorfer: Should create smoother fonts
\usepackage{ae,aecompl}

\addtolength{\textwidth}{2cm}
\addtolength{\textheight}{2cm}
\addtolength{\oddsidemargin}{-1.0cm}
\addtolength{\evensidemargin}{-1.0cm}
\addtolength{\topmargin}{-1.5cm}

%% No Serifs: Put comment markers in front of the next three lines otherwise
\renewcommand{\ttdefault}{cmtt}
\renewcommand{\rmdefault}{phv}  % Helvetica for roman type as well as sf
\renewcommand{\ttdefault}{pcr}  % use Courier for fixed pitch, if needed

\newcommand{\?}{\discretionary{/}{}{/}}
\newcommand{\liter}[0]{/home/ruf/Lib/Bibl/}
\newcommand{\fref}[1]{\mbox{Figur~\ref{#1}}}

\pagestyle{fancy}
%%-lpr Note: 'chapters' are defined for 'book's only
%%-lpr       in articles, we make use of sections only
%%-lpr \renewcommand{\chaptermark}[1]{\markboth{#1}{}}
\renewcommand{\sectionmark}[1]{\markright{\thesection\ #1}}
\fancyhf{}
\fancyhead[LE,RO]{\bfseries\thepage}
\fancyhead[LO]{\bfseries\rightmark}
\fancyhead[RE]{\bfseries\leftmark}
\renewcommand{\headrulewidth}{0.5pt}
\addtolength{\headheight}{0.5pt}
\fancypagestyle{plain}{%
   \fancyhf{}
   \fancyfoot[C]{\bfseries \thepage}
   \fancyhead{}%get rid of headers on plain pages
   \renewcommand{\headrulewidth}{0pt} % an the line
}
\newcommand{\clearemptydoublepage}{\newpage{\pagestyle{empty}\cleardoublepage}}

\setlength{\parindent}{0in}

\hyphenation{Lukas not-to-hyphen-else-where}

\newcommand{\Appendix}[2][?]
{
  \refstepcounter{section}
  \addcontentsline{toc}{appendix}
  {
    \protect\numberline{\appendixname~\thesection} %1
  }
  {
    \flushright\large\bfseries\appendixname\ \thesection\par
    \nohypens\centering#1\par
  }
  \vspace{\baselineskip}
}

\let\margin\marginpar
\newcommand\myMargin[1]{\margin{\raggedright\scriptsize #1}}
\renewcommand{\marginpar}[1]{\myMargin{#1}}

\newcommand\CHECK{\myMargin{CHECK}}
\newcommand\NEW{\myMargin{NEW}}

%% adapt headers %%
% from http://www.markschenk.com/tensegrity/latexexplanation.html
% Result:
% - No headers on empty pages before new chapter
% - To avoid header on other pages (e.g. in the abstract), set pagestlye to plain
\makeatletter
\def\cleardoublepage{\clearpage\if@twoside \ifodd\c@page\else
    \hbox{}
    \thispagestyle{plain}
    \newpage
    \if@twocolumn\hbox{}\newpage\fi\fi\fi}
\makeatother

%% allow to set a watermark %%
% from http://www.goodcomputingtips.com/site/2010/06/how-to-insert-watermark-in-latexpdflatex-documents/
% - to include a watermark, define \watermark in the main document, e.g.: \newcommand{\watermark}{MY WATERMARK TXT}
\ifdefined\watermark
  \usepackage{graphicx,type1cm,eso-pic,color}
  \makeatletter
            \AddToShipoutPicture{
              \setlength{\@tempdimb}{.5\paperwidth}
              \setlength{\@tempdimc}{.5\paperheight}
              \setlength{\unitlength}{1pt}
              \put(\strip@pt\@tempdimb,\strip@pt\@tempdimc){
          \makebox(0,0){\rotatebox{55}{\textcolor[gray]{0.95}
          {\fontsize{5cm}{5cm}\selectfont{\watermark}}}}
              }
          }
  \makeatother
\fi


\newcommand{\textRightArrow}[0]{$\,\to\,$}
%% Definition style
\newcommand{\definition}[1]{\par \textbf{#1:}}

%% Integral from -infty to infty
\newcommand{\totint}[1]{\int_{-\infty}^{\infty}#1}

%% enable circuits drawing
\usepackage{circuitikz}

% make table of contents clickable
\usepackage{hyperref}

\newcommand{\uU}{\underline{U}}
\newcommand{\uI}{\underline{I}}
\newcommand{\uZ}{\underline{Z}}

% Klammern
\newcommand{\klammern}[1]{\left( #1 \right)}
\newcommand{\eckigeKlammern}[1]{\left[ #1 \right]}
\newcommand{\geschwungeneKlammern}[1]{\left\{ #1 \right\}}
\newcommand{\innerProduct}[2]{\left< #1 , #2 \right>}
\newcommand{\evaluated}[1]{|_{#1}}

% math. Funktionen mit klammern
\newcommand{\abs}[1]{\left| #1 \right|}
\newcommand{\floor}[1]{\left\lfloor #1 \right\rfloor}
\newcommand{\ceil}[1]{\left\lceil #1 \right\rceil}

\newcommand{\twoNorm}[1]{\left| \abs{#1} \right|_2}
\newcommand{\energy}[1]{\twoNorm{#1}^2}

\newcommand{\condApprox}[1]{\overset{#1}{\approx}}
\newcommand{\erfc}[1]{erfc\left(#1\right)}
\newcommand{\norm}[1]{\left|\left| #1 \right|\right|}
\newcommand{\case}[1]{\left\{\begin{array}{cl} #1 \end{array}\right.}
\renewcommand{\cos}[1]{\mbox{cos}\left(#1\right)}
\renewcommand{\tan}[1]{\mbox{tan}\left(#1\right)}
\renewcommand{\sin}[1]{\mbox{sin}\left(#1\right)}
\newcommand{\sinc}[1]{\mbox{sinc}\left(#1\right)}
\newcommand{\arcosh}[1]{\mbox{arcosh}\left(#1\right)}
\newcommand{\arsinh}[1]{\mbox{arsinh}\left(#1\right)}
\renewcommand{\tanh}[1]{\mbox{tanh}\left(#1\right)}
\renewcommand{\cosh}[1]{\mbox{cosh}\left(#1\right)}
\renewcommand{\Re}[1]{\mbox{Re}\geschwungeneKlammern{#1}}
\renewcommand{\Im}[1]{\mbox{Im}\geschwungeneKlammern{#1}}
\newcommand{\E}[1]{\mbox{E}\eckigeKlammern{#1}}
\newcommand{\equalDistribution}[0]{\overset{\mathcal{L}}{=}}
\newcommand{\Var}[1]{\mbox{Var} \eckigeKlammern{#1}}
\newcommand{\Cov}[1]{\mbox{Cov} \eckigeKlammern{#1}}
\newcommand{\myint}[4]{\int_{#1}^{#2}{#3 \d{#4}}}
\newcommand{\argmax}[2]{\underset{#1}{\mbox{argmax}}\geschwungeneKlammern{#2}}
\newcommand{\argmin}[2]{\underset{#1}{\mbox{argmin}}\geschwungeneKlammern{#2}}



%define math spaces
\newcommand{\R}[0]{\mathbb{R}}
\newcommand{\C}[0]{\mathbb{C}}
\newcommand{\N}[0]{\mathbb{N}}
\newcommand{\Z}[0]{\mathbb{Z}}



%d in integrals
\renewcommand{\d}[1]{\mbox{ d#1}}

%sonstige funktionen
\newcommand{\complexConj}[1]{#1^*}


%listings
\usepackage{listings}



